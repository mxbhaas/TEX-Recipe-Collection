\documentclass{report}
\usepackage[margin=2cm]{geometry}
\usepackage{imakeidx}
\usepackage{darkmode}
\usepackage{tikzpagenodes}
\usepackage{eso-pic}
\usepackage[toc]{multitoc}
\usepackage{nameref}
\usepackage[hidelinks]{hyperref}
 
\renewcommand*{\multicolumntoc}{2}
\setlength{\columnseprule}{0pt}
\setlength\columnsep{1.5cm}

\hypersetup{
  pdftitle={Private Recipe Collection 2024},
  pdfauthor={Maximilian B. Haas},
}

\AddToShipoutPictureBG{\begin{tikzpicture}[overlay,remember picture]
 \fill[teal,even odd rule] (current page.south west)
 rectangle (current page.north east)
 ([xshift=-1.7cm,yshift=-1.7cm]current page text area.south west) 
 rectangle ([xshift=1.7cm,yshift=1.7cm]current page text area.north east);
\end{tikzpicture}}

\title{Private Recipe Collection}
\date{2024}
\author{Maximilian B. Haas \& Adriana Vitvitchi}

\setcounter{secnumdepth}{0}
\makeindex[intoc]
\preto\endtheindex{\par\nobreak\noindent\hfill}

\begin{document}
\begin{titlepage}
	\centering
	\vspace*{5cm}
	{\huge\bfseries PRIVATE RECIPE COLLECTION\par}
	\vspace{1cm}
	{\Large\itshape M. B. H. \& A. V.\par}
	\vfill
	{\Large Coming out of an obsession with cooking and -- even more -- eating, this recipe collection was initially created in April 2024 and is being maintained since then.}
	\vspace*{1cm}
\end{titlepage}

\pagenumbering{gobble}
\tableofcontents
\pagebreak

\pagenumbering{arabic}
\part{Starters and Fingerfood}

\chapter{Starter Dishes}
\section{Salmon Tatar}
\index{Salmon Tatar}
\index{Tatar}
\textit{Source: Cook Book "Der große Lafer" by Johann Lafer}\\
\begin{minipage}[t]{0.21\textwidth} \vspace{0pt}
300g fresh salmon\\
300g smoked salmon\\
3 spring onions\\
1 garlic clove\\
1 lime\\
honey\\
2 tbsp olive oil\\
150g crème fraîche
\end{minipage}
\hfill
\begin{minipage}[t]{0.78\textwidth} \vspace{0pt}
	\begin{enumerate}
		\item Chop both salmon types in small cubes. Add the finely chopped garlic and spring onion, half of the lime juice, the olive oil and the honey. Put it in the fridge to rest for around 30min.
		\item For the cream, mix the crème fraîche with the skin of half the lime, the remaining lime juice, salt and pepper.
		\item Assemble the tatar with a metal ring, and add the cream on top.
	\end{enumerate}	
\end{minipage}

\chapter{Fingerfood}
\section{Hummus a la Turka}
\index{Hummus}
\textit{Source: Greek/Turkish cooking class}\\
\begin{minipage}[t]{0.21\textwidth} \vspace{0pt} 
240g chickpeas\\
4 tbsp olive oil\\
20ml lemon juice\\
30g sesame seeds\\
3 tbsp tahini\\
1 onion\\
2 garlic cloves\\
3 tbsp greek yoghurt\\
1/2 tsp cumin\\
1/2 tsp paprika powder
\end{minipage}
\hfill
\begin{minipage}[t]{0.78\textwidth} \vspace{0pt}
	\begin{enumerate}
		\item Rinse the chickpeas with water through a strainer and let them dry. Peel the onion and the garlic, and chop it in fine dices. Fry it in a little bit of olive oil.
		\item Add chickpeas, olive oil, lemon juice, tahini and the fried onion garlic mixture in a blender and blend it to a fine paste, add 2-3 tbsp of water if needed.
		\item Mix in the yoghurt, and taste with salt, cumin and paprika powder.
		\item Fry the sesame in a pan without oil, and let it cool. Then add it on top of the hummus.
	\end{enumerate}	
\end{minipage}

\section{Tzatziki}
\index{Tzatziki}
\textit{Source: Greek/Turkish cooking class}\\
\begin{minipage}[t]{0.21\textwidth} \vspace{0pt} 
500g greek yoghurt\\
1/2 cucumber\\
3 garlic cloves\\
10ml olive oil\\
1 lemon\\
\end{minipage}
\hfill
\begin{minipage}[t]{0.78\textwidth} \vspace{0pt}
	\begin{enumerate}
		\item Cut the cucumber in half, remove the seeds from the center and shred on a grater. 
		\item Place the grated cucumber in a sieve and add salt so that the cucumber loses water. Then leave to drain in the sieve for approx. 30min, wring out again and place in a bowl. 
		\item Stir in the yoghurt, finely chop the garlic and add with olive oil and salt. olive oil and salt. Mix in pepper and lemon juice and season to taste again.
	\end{enumerate}	
\end{minipage}

\section{Melitzanossalata (Eggplant Purée)}
\index{Melitzanossalata}
\index{Eggplant Purée}
\textit{Source: Greek/Turkish cooking class}\\
\begin{minipage}[t]{0.21\textwidth} \vspace{0pt} 
1kg eggplants\\
2 garlic cloves\\
1 bunch parsley\\
60ml olive oil\\
white wine vinegar\\
\end{minipage}
\hfill
\begin{minipage}[t]{0.78\textwidth} \vspace{0pt}
	\begin{enumerate}
		\item Preheat the oven to 180°C. Wash the eggplant, dry well and place on a rack in the oven. Bake until the skin looks a little burnt and the inside is very soft. 
		\item Remove the eggplants from the oven, rinse with cold water and peel off the skin immediately. Now cut the eggplants into small pieces, place in a bowl and   mix with the crushed garlic cloves. 
		\item Season with salt and stir in the oil drop by drop. Stirring constantly, add white wine vinegar or lemon juice to taste.  Finally, add the chopped parsley.
	\end{enumerate}	
\end{minipage}

\section{Dolmades (Filled Vine Leaves)}
\index{Dolmades}
\index{Filled Vine Leaves}
\textit{Source: Greek/Turkish cooking class}\\
\begin{minipage}[t]{0.21\textwidth} \vspace{0pt} 
200g round-grain rice\\
300g pickled vine leaves\\
4 tbsp raisins\\
200g onion\\
1/2 bunch parsley\\
1/2 bunch dill\\
2 twigs mint\\
5 tbsp olive oil\\
40g pine nuts\\
1 lemon\\
\end{minipage}
\hfill
\begin{minipage}[t]{0.78\textwidth} \vspace{0pt}
	\begin{enumerate}
		\item Wash the rice and place it in a pot. Cover with water and parboil for about 15min. Briefly dip the vine leaves in boiling water, then spread them out on a kitchen towel to drain and trim the stems.
        \item Drain the rice and place it in a large bowl. Soak the raisins in lukewarm water for about 30min. Peel and finely dice the onions. Add the drained raisins, chopped herbs, and onions to the rice. Add the pine nuts, half of the lemon juice, and 2 tbsp of olive oil to the bowl and mix well. Season with salt and pepper.
        \item To fill each vine leaf, place about 1 tablespoon of the filling on a leaf, fold in the sides, and roll from the stem end to the tip. Continue until all vine leaves are filled this way. Place all the rolls tightly together in a pot, drizzle with the remaining lemon juice and olive oil.
        \item Place a plate upside down on top to weigh down the rolls and pour enough hot water to cover them all. Simmer over low heat for about 1h, then let cool in the pot.    
	\end{enumerate}	
\end{minipage}

\part{Main Dishes}

\chapter{Indian}
\section[Butter Chicken]{Butter Chicken - Murgh Makhani}
\index{Butter Chicken}
\index{Murgh Makhani}
\textit{Source: TBD}\\
\begin{minipage}[t]{0.21\textwidth} \vspace{0pt} 
\textbf{Meat \& Marination}\\
400g chicken \\
2 tsp ginger garlic paste \\
1/2 tsp turmeric\\
2 tsp red chili powder\\
1 tsp coriander  \\
1 tsp cumin  \\
1 tsp chaat masala  \\
1 tsp garam masala \\
1 tbsp lemon juice \\

\textbf{For the Sauce}\\
1 tsp cumin seeds \\
1 pod cardamom\\
4cm ginger \\
8 garlic cloves\\
1 tsp red chili powder\\
1 tsp kasori methi\\
15 -20 cashew nuts\\
2 green chilli\\
2 big tomatoes\\
1/2 tsp turmeric
\end{minipage}
\hfill
\begin{minipage}[t]{0.79\textwidth}
	\begin{enumerate}
		\item Cut the chicken into bite-sized pieces and mix with all the ingredients from the marinade. Set aside in the fridge and let it marinade for at least 2h. Soak the cashew nuts in water for around 15min, then drain.
		\item In a big pot, head some oil and then add the cumin and cardamom, fry until fragrant. Then add the roughly chopped onion, the garlic cloves and the ginger. Stir in all the remaining ingredients for the sauce, and let it simmer for around 10min. Take it off the stove and let it cool down.
		\item Add the sauce into a blender and mix until very fine. Meanwhile, add some oil into the same pot and fry the marinated chicken pieces until golden brown. Take them out of the pan, and then add the blended sauce through a fine sieve, to filter out the tomato skin and whole spices. Add some butter, and let it simmer for another 15min.
		\item Finally, add a tbsp of sugar, cream and the chicken into the masala. Season with salt and garnish with crushed kasori methi, and service with naan and rice.
	\end{enumerate}	
\end{minipage}

\section{Punjabi Palak Paneer}
\index{Palak Paneer}
\textit{Source: Cook book "India: The Cookbook" by Pushpesh Pant}\\
\begin{minipage}[t]{0.21\textwidth} \vspace{0pt} 
500g washed spinach \\
400g paneer\\
2 green chilli\\
3cm chopped ginger\\
4 garlic cloves\\
2 large red onions\\
2 chopped tomatoes
\begin{flushleft}
1 tsp each of cumin seeds, turmeric powder, red chilli, cumin, coriander, garam masala\\ 
\end{flushleft}
4 tbsp fresh cream
\end{minipage}
\hfill
\begin{minipage}[t]{0.78\textwidth} \vspace{0pt}
	\begin{enumerate}
		\item Cook spinach, blanch and put in blender. Heat oil in a pan, add cumin seeds, fry it for few seconds then add green chilli, ginger, garlic. Add onions and fry it on medium flame till it turns light brown (2-3 minutes) . 
		\item Add chopped tomato, cook on medium flame until they turn  soft (2-3 minutes). Let cool and add to blend. Add a bit of water and blend it into smooth paste. 
		\item Heat oil in same pan, add turmeric, red chilli, coriander and cumin powder. Fry it, add the prepared paste and keep stirring. Cover the pan and cook it for 2-3 minutes on medium flame. 
		\item Add fresh cream and paneer cubes, then add garam masala and salt. Cover the pan and cook for 2 minutes on low flame, turn off the flame and serve.
	\end{enumerate}	
\end{minipage}

\section{Mutton Rogan Josh}
\index{Rogan Josh}
\index{Indian Lamb Curry}
\textit{Source: YouTube}\\
\begin{minipage}[t]{0.21\textwidth} \vspace{0pt}
1kg lamb shoulder\\
1/2 tsp turmeric\\
1 tsp fennel seeds\\
8 cloves\\
2 black cardamom\\
handful of red chillies\\
4 tbsp mustard oil\\
1 tsp asafoetida\\
4 green cardamom\\
2 sticks cinnamon\\
1 tsp cumin seeds\\
50g curd
\end{minipage}
\hfill
\begin{minipage}[t]{0.78\textwidth} \vspace{0pt}
	\begin{enumerate}
		\item Chop the lamb in 5cm cubes and put in a pressure cooker. Add salt, turmeric, fennel seeds, cloves, cardamom and water until the meat is covered, and simmer until it is around 80\% cooked.
		\item Boil the dry red chillies in water, sear them in oil and make a smooth paste in the blender.
		\item Heat mustard oil in a big pan. Add asafoetida, cardamom, cinnamon, cloves, cumin and the drained boiled mutton. Then add the red chilli paste, and stir until dissolved. Add the curd, stock and salt and sugar to taste, and let simmer for another 5min. To finish off, add some (clarified) butter, and serve with rice or naan.
	\end{enumerate}	
\end{minipage}

\chapter{Mediterranean}
\section{Achnista Mydia (Steamed Mussels)}
\index{Achnista Mydia}
\index{Mussels in Ouzo Sauce}
\textit{Source: Greek/Turkish cooking class}\\
\begin{minipage}[t]{0.21\textwidth} \vspace{0pt} 
1.8kg mussels\\
70g celeriac \\
70g carrots \\
70g leek\\
1 red onion\\
1 garlic clove\\
olive oil \\
100ml ouzo \\
1 bunch parsley
\end{minipage}
\hfill
\begin{minipage}[t]{0.78\textwidth} \vspace{0pt}
	\begin{enumerate}
		\item Wash the mussels thoroughly in cold water, removing any beards (the fibrous threads) if present. Discard any mussels that are open as they may be spoiled or dead. Wash or peel the specified vegetables and finely dice them. Lightly crush the garlic clove.
   	 	\item Heat a large pot until very hot, add a little olive oil, and then immediately pour in the mussels. Cover and cook for about a minute.
    	\item Add the vegetables and garlic with a pinch of salt. Stir well and de-glaze with Ouzo. Cover and cook the mussels for about 2-3min. Meanwhile, pick and finely chop the parsley. Add the parsley to the pot and adjust the seasoning with salt and pepper if necessary.
    	\item Divide the mussels onto plates and drizzle with some of the cooking liquid. Finally, drizzle with a little olive oil.
	\end{enumerate}	
\end{minipage}

\section{Stifado (Braised Beef with Onions)}
\index{Stifado}
\index{Braised Beed with Onions}
\textit{Source: Greek/Turkish cooking class}\\
\begin{minipage}[t]{0.21\textwidth} \vspace{0pt} 
1.5kg beef shoulder \\
75g butter \\
750g diced tomatoes \\
1 kg onions \\
500g pearl onions \\
400ml red liqueur wine \\
2 bay leaves \\
2 cinnamon sticks \\
2 tsp paprika powder 
\end{minipage}
\hfill
\begin{minipage}[t]{0.78\textwidth} \vspace{0pt}
	\begin{enumerate}
		\item Roughly chop the meat into cubes. Melt the butter in a pot and brown the meat cubes on all sides. Add the diced tomatoes, bring to a brief boil, then add the onions. Sauté for a few minutes, then pour in the wine. 
		\item Season with bay leaves, cinnamon, paprika powder, salt, and pepper. Add enough water to cover everything well. Cover the pot, reduce the heat, and let it simmer for about 1h, or until the meat is tender. Check occasionally to see if more water needs to be added.
    	\item Once the meat is tender and the sauce has thickened, remove the pot from the heat and serve on plates.
	\end{enumerate}	
\end{minipage}

\section{Shakshouka}
\index{Shakshouka}
\textit{Source: Kitchen Stories}\\
\begin{minipage}[t]{0.21\textwidth} \vspace{0pt}
5 tomatoes\\
1 onion\\
2 garlic cloves\\
5cm ginger\\
1 chilli\\
2 peppers (red \& yellow)\\
1 tsp paprika powder\\
1 tsp cumin powder\\
800g canned tomatoes\\
50ml vegetable broth\\
4 eggs\\
50g feta cheese
\end{minipage}
\hfill
\begin{minipage}[t]{0.78\textwidth} \vspace{0pt}
	\begin{enumerate}
		\item Finely dice the onions, garlic, ginger and chilli. Cut the bell peppers in thin slices and then cop, and the fresh tomatoes in cubes.
		\item Heat oil in a big pan and add the onions, garlic, chilli and bell pepper to fry for around 5min. Then add the ginger, the paprika powder and cumin powder to sear for another few minutes until fragrant. De-glaze with the canned tomatoes and add the fresh ones as well. Season with salt and pepper and leave a bit to simmer.
		\item Preheat the oven to 190°C without air. Add the broth to the shakshouka, and mix everything again. Make small moulds for the eggs, and drop them into the pan. Add feta cheese on the top and bake in the oven for 7-10min. Garnish with chopped parsley and serve.
	\end{enumerate}	
\end{minipage}

\section{Greek Tomato Feta Soup}
\index{Tomato Feta Soup}
\textit{Source: Kitchen Stories}\\
\begin{minipage}[t]{0.21\textwidth} \vspace{0pt}
200g orzo\\
400g chopped tomatoes\\
150g feta cheese\\
2 garlic cloves\\
1 onion\\
4 tbsp olive oil\\
1l chicken broth\\
3 tbsp tomato paste\\
1 lemon
\end{minipage}
\hfill
\begin{minipage}[t]{0.78\textwidth} \vspace{0pt}
	\begin{enumerate}
		\item Finely chop garlic and onion. Heat olive oil in a big pot, and add garlic and onion until fragrant. Add the tomato paste and fry, before adding oregano and chilli flakes.
		\item Add the canned tomatoes, juice of the lemon and the broth to the pot. Season with salt, pepper and sugar and bring the soup to a boil. Add the orzo and cook for 10min, while stirring now and then.
		\item Season the soup again, and add the crumbled feta on top of the soup when serving.
	\end{enumerate}	
\end{minipage}

\section{Braised Chicken with Spinach}
\index{Chicken}
\textit{Source: Kitchen Stories}\\
\begin{minipage}[t]{0.21\textwidth} \vspace{0pt} 
1kg chicken thighs\\
200g spinach\\
80g sun-dried tomatoes\\
2 garlic cloves\\
1 lemon\\
1 tbsp unsalted butter\\
200ml chicken stock\\
200ml heavy cream\\
2 tbsp cornstarch\\
salt\\
pepper\\
olive oil
\end{minipage}
\hfill
\begin{minipage}[t]{0.78\textwidth} \vspace{0pt}
	\begin{enumerate}
		\item Peel and finely chop the garlic, zest the lemon and slice the sun-dried tomatoes into thin strips.
		\item Heat a little olive oil in a large ovenproof pot and brown the chicken thighs about 4 minutes per side until golden. Season with salt and pepper and set aside.
		\item In the same pot melt the butter and sauté the garlic for ~2 minutes. Pour in the chicken stock to deglaze. In a small bowl whisk the cream with the cornstarch, then stir into the pot and simmer uncovered 5 minutes until the sauce thickens.
		\item Preheat the oven to 160°C. Return the chicken to the pot, sprinkle over the lemon zest, cover and braise in the oven for about 50 minutes.
		\item Remove the pot, add the sun-dried tomatoes and continue to braise 10 minutes. Stir in the spinach and cook a further 5 minutes until wilted. Serve the braised chicken with pasta, rice or bread.
	\end{enumerate}	
\end{minipage}

\chapter{Eastern European}
\section[Plăcintă]{Plăcintă Moldovnesti}
\index{Plăcintă}
\textit{Source: YouTube}\\
\begin{minipage}[t]{0.21\textwidth} \vspace{0pt} 
400ml lukewarm water\\
1 tsp salt\\
1 tbsp sugar\\
1 tbsp vinegar\\
1 tbsp sunflower oil\\
600g flour\\
600g cottage cheese (9\% fat)\\
dill\\
spring onion
\end{minipage}
\hfill
\begin{minipage}[t]{0.78\textwidth} \vspace{0pt}
	\begin{enumerate}
		\item In a big bowl, add the water, salt, sugar, vinegar and oil. Then gradually add the sifted flour to form a consistent dough. Cover it with plastic wrap and let it rest for 30min.
		\item For the filling, crumble the cottage cheese in a bowl, and finely chop the dill and spring onion. Mix everything together and season with salt and pepper.
		\item Divide the dough into about 10 equal pieces and roll them out to a thin circle (roughly 30cm). Then add the filling in the middle, and fold the edges over to form a round shape. Press the folds tight so the filling doesn't leak out.
		\item Heat oil in a pan, and fry the placintas on both sides until golden brown. Drain on kitchen towel and serve warm.
	\end{enumerate}	
\end{minipage}

\chapter{Thai}
\section{Red Thai Curry}
\index{Red Thai Curry}
\textit{Source: TBD}\\
\begin{minipage}[t]{0.21\textwidth} \vspace{0pt} 
1.5 tbsp curry paste\\
150g meat\\
300ml coconut milk\\
2 tsp fish sauce\\
1 tbsp sugar\\
1/4 cup basil\\
bell pepper\\
green beans\\
3-4 kaffir lime leaves\\
120ml chicken broth\\
a handful thai basil
\end{minipage}
\hfill
\begin{minipage}[t]{0.78\textwidth} \vspace{0pt}
	\begin{enumerate}
		\item Heat oil in saucepan, add curry paste and cook 1 minute on medium. Add half of the coconut milk and stir to combine, then bring to a boil.
		\item Add meat and the broth. Stir to cook and let simmer for 10 minutes or until meat is tender.
		\item Add kaffir leaves, sugar, and fish sauce. Add the remaining coconut milk and stir to combine.
		\item Add veggies and stir. Bring to a boil and cook for a few minutes, then turn off the heat. Add the Thai basil and serve.
	\end{enumerate}	
\end{minipage}

\chapter{French}
\section{Bœuf en Daube à la Provençale}
\index{Bœuf en Daube à la Provençale}
\textit{Source: Cook book "Provence - The Cookbook" by Caroline Craig}\\
\begin{minipage}[t]{0.21\textwidth} \vspace{0pt} 
\textbf{Marinade:}\\
2kg beef shoulder\\
1 orange\\
5 carrots\\
1 bouquet garni\\
1 bay leaf\\
3 cloves\\
13 white peppercorns\\
5 juniper berries\\
1l red wine\\
133ml red vinegar\\

\textbf{Sauce:}\\
1 large onion\\
266g bacon\\
4 garlic cloves\\
1 orange\\
400ml beef stock\\
4 tbsp olive oil\\
1 half cinnamon stick\\
400g black olives
\end{minipage}
\hfill
\begin{minipage}[t]{0.78\textwidth} \vspace{0pt}
	\begin{enumerate}
		\item Cut meat in pieces. Chop the carrots, zest the orange and put some pieces aside. Cut the orange and onion in slices, add to bouquet garni, the cloves, the juniper, some muscat, the crushed peppercorns and the red wine with the vinegar overnight in the fridge.
		\item Next day, preheat oven to 150°. Take meat out of marinade and dry. Cut the bacon and the onions in rings. Sear meat in batches in oil in a big pot, then add the bacon and onion. Put all meat back in the pot and mix, then add salt and pepper to taste.
		\item Reduce marinade by a third without vegetables in a second pot. Then add with vegetables to the meat, add all other spices including the orange zest, garlic and cinnamon. Add beef broth until everything is covered and let simmer for 5h in the oven.
		\item Take meat and vegetables out and put sauce through a sieve to filter all pieces out. Thicken the sauce with starch, add meat and vegetables back and also the olives. Add salt \& pepper to taste and keep warm until served.
	\end{enumerate}	
\end{minipage}

\section[Spinach Salmon Quiche]{Spinach Salmon Quiche (for 28cm tarte pan)}
\index{Quiche}
\index{Spinach Salmon Quiche}
\textit{Source: self-created}\\
\begin{minipage}[t]{0.21\textwidth} \vspace{0pt} 
1 pâte brisée\\
400g spinach\\
200g smoked salmon\\
1 shallot\\
1 garlic clove\\
1 egg\\
1 egg yolk\\
200ml crème liquide\\
200ml crème fraîche 
\end{minipage}
\hfill
\begin{minipage}[t]{0.78\textwidth} \vspace{0pt}
	\begin{enumerate}
		\item Prepare the quiche dough for a 28cm tarte.
		\item Cook the onions, garlic and the spinach until all water evaporated, then season with salt and pepper. Prep the salmon by cutting it in small cubes.
		\item Preheat the oven to 180°C with air. Whisk the egg and the both crèmes together, and season with salt and pepper.\\
		\textit{Optional: You can also add some herbs, nutmeg or parmesan cheese.}
		\item Roll out the quiche dough and drape it in the tarte pan. Take a fork and poke holes in the bottom of the dough. Then add the salmon, the spinach and the crème on top. Place the quiche in the oven and bake for around 40min.
	\end{enumerate}	
\end{minipage}

\section{Coq au Vin}
\index{Coq au Vin}
\index{Red Wine Chicken}
\textit{Source: Kitchen Stories}\\
\begin{minipage}[t]{0.21\textwidth} \vspace{0pt}
4 chicken legs\\
6 bacon strips\\
200g champignons\\
10 shallots\\
400g potatoes\\
4 carrots\\
2 tbsp butter\\
2 tbsp tomato paste\\
750ml red wine\\
2 bay leaves\\
4 twigs thyme\\
2 twigs rosemary\\
4 tbsp corn starch
\end{minipage}
\hfill
\begin{minipage}[t]{0.78\textwidth} \vspace{0pt}
	\begin{enumerate}
		\item Separate the chicken legs with a knife into drumsticks and shoulder. Slice the bacon in thin strips, halve the champignons, peel the shallots and cut them in quarters. Peel the potatoes and cut into 2cm cubes, peel the carrots as well and cut into 1cm thick slices.
		\item Add oil into a pan, salt the chicken pieces from both sides and sear them until crispy brown on both sides. Then set the chicken aside on a plate. In the same pan, add the butter, bacon and all the veggies and sear for 2-3min while stirring, then season with salt and pepper. Add the tomato paste, and then de-glaze with half of the red wine. Boil for around 5min, and let the liquid reduce.
		\item Add the remaining wine and then chicken broth. Place chicken pieces in the pot again, and add the bouquet garni (bay leaves, thyme and rosemary). Close the lid and let it simmer for 30min.
		\item Mix the corn starch with some cold water in a separate small bowl, and then add to the coq au vin to thicken the sauce. Season to taste with salt and pepper, and serve with chopped parsley.
	\end{enumerate}	
\end{minipage}

\chapter{Italian}
\section{Pasta al limone}
\index{Pasta}
\index{Pasta al limone}
\textit{Source: Kitchen Stories}\\
\begin{minipage}[t]{0.21\textwidth} \vspace{0pt}
400g tagliatelle\\
1 lemon\\
1 red chilli\\
1 garlic clove\\
50g parmesan cheese\\
50g butter\\
50ml olive oil\\
20g parsley\\
600ml pasta water
\end{minipage}
\hfill
\begin{minipage}[t]{0.78\textwidth} \vspace{0pt}
	\begin{enumerate}
		\item Do the prep by crushing the garlic, zesting the lemon, grating the parmesan, finely slicing the chilli and mincing the parsley.
		\item In a frying pan, add butter and olive oil, and then the chilli and garlic. In another pot, start cooking the pasta. Right before finishing, take out the pasta water for the sauce.
		\item To the pan, add lemon juice, the zest and the pasta water. Let the sauce reduce and season with salt and pepper. Then add the strained pasta and the parmesan, and garnish with the parsley.
	\end{enumerate}	
\end{minipage}

\section{Penne alla Vodka}
\index{Pasta}
\index{Penne alla Vodka}
\textit{Source: TBD}\\
\begin{minipage}[t]{0.21\textwidth} \vspace{0pt}
400g pasta\\
80g prosciutto\\
3 garlic cloves\\
150ml cream\\
4 tbsp vodka\\
400g tomato purée\\
3 basil stems
\end{minipage}
\hfill
\begin{minipage}[t]{0.78\textwidth} \vspace{0pt}
	\begin{enumerate}
		\item Chop the garlic in thin slices, and the prosciutto in cubes. Head olive oil in a pan, and fry the prosciutto until lightly brown. Then add the garlic slices until fragrant.
		\item De-glaze with the vodka, and then add the tomato purée and the basil. Let cook until the sauce has reduced. Then add the cream to the sauce until it becomes the desired color, and season with salt and pepper.
		\item Cook the pasta in the meantime, and right before its al dente, add some pasta water to the tomato sauce. Drain the pasta, add to the sauce in the pan and serve.
	\end{enumerate}	
\end{minipage}

\chapter{Other}
\section{Salmon with Spinach and Mushrooms}
\index{Salmon}
\index{Salmon with spinach and mushrooms}
\textit{Source: Kitchen Stories}\\
\begin{minipage}[t]{0.21\textwidth} \vspace{0pt} 
560g salmon with skin\\
300g baby spinach\\
350g mushrooms\\
1 lemon\\
2½ tbsp parsley\\
1 tbsp chives\\
1 garlic clove\\
2 tbsp oil\\
2 tbsp butter\\
50ml milk
\end{minipage}
\hfill
\begin{minipage}[t]{0.78\textwidth} \vspace{0pt}
	\begin{enumerate}
		\item Season the salmon with salt and pepper, quarter the mushrooms and chop the chives and parsley. Chop the lemon and take half of its juice.
		\item Sear the salmon with oil and butter 5min each side.
		\item Put the mushrooms and the garlic in the same pan, add the flour after the mushrooms are browned and continue cooking for 1min.
		\item Add the milk to the pan and taste it with salt, pepper and nutmeg. Then add the spinach and let it become soft in the sauce. Add parsley and lemon juice. Put salmon on top and serve.
	\end{enumerate}	
\end{minipage}

\part{Desserts}

\chapter{Cold Desserts}
\section{Tiramisu}
\index{Tiramisu}
\textit{Source: TBD}\\
\begin{minipage}[t]{0.21\textwidth} \vspace{0pt} 
3 eggs\\
50g sugar\\
500g mascarpone\\
150ml espresso\\
30ml amaretto\\
30 ladyfinger biscuits
\end{minipage}
\hfill
\begin{minipage}[t]{0.78\textwidth} \vspace{0pt}
	\begin{enumerate}
		\item Temper and beat the egg yolks with half of sugar until thick, then add the mascarpone.
		\item Whip the egg whites with salt and slowly add the remaining sugar until it forms stiff peaks. Add to the mixture from Step 1.
		\item Mix espresso and amaretto together, then brush/dip the ladyfingers with the mixture after spreading them out in the form.
		\item Finally, layer the tiramisu alternating with the soaked ladyfingers and the mascarpone cream. Right before serving, add a layer of cocoa dust on top of the cream layer above.
	\end{enumerate}	
\end{minipage}

\section[Vanilla Crème Pâtissière]{Vanilla Crème Pâtissière (for 28cm tarte pan)}
\index{Vanilla Crème Pâtissière}
\index{Tarte}
\textit{Source: Online, La Pâticesse}\\
\begin{minipage}[t]{0.21\textwidth} \vspace{0pt}
500g milk\\
90g sugar\\
1 egg\\
3 egg yolks\\
1-2 vanilla beans\\
25g flour\\
25g corn starch
\end{minipage}
\hfill
\begin{minipage}[t]{0.78\textwidth} \vspace{0pt}
	\begin{enumerate}
		\item Bring the milk to a boil with the vanilla beans and 20g sugar. Above the same pot, set up a bowl as water bath with the egg, the yolks, the remaining sugar and the starch, and temper it from the steam over the milk pot.
		\item Slowly add the hot milk to the tempered egg mixture. Once all is added, remove the vanilla bean and keep simmering until the mixture is thickened.
		\item Spread on a foil-lined baking tray and cover with plastic wrap in direct contact with the custard, so the cream forms no skin, and let it cool down.
	\end{enumerate}	
\end{minipage}

\chapter{Warm Desserts}

\part{Baking}
\chapter{Basic Doughs}

\section{Garlic Naan}
\index{Naan}
\index{Garlic Naan}
\begin{minipage}[t]{0.21\textwidth} \vspace{0pt} 
130g warm water\\
6g sugar\\
1 pack dry yeast\\
2 tbsp yoghurt\\
300g flour (type 550)\\
7g salt\\
1 tbsp butter
\end{minipage}
\hfill
\begin{minipage}[t]{0.78\textwidth} \vspace{0pt}

	\begin{enumerate}
		\item Combine and mix all dry ingredients.
		\item Add water, sugar and yoghurt and knead to a smooth dough, then let it rest for 1h.
		\item Split in 6 pieces, roll out and bake in a hot pan without fat about 3min on each side until bubbles start to form.
		\item Brush with (garlic) butter and serve.
	\end{enumerate}	
\end{minipage}

\section{Pizza Dough}
\index{Pizza Dough}
\begin{minipage}[t]{0.21\textwidth} \vspace{0pt} 
570g typo 00\\
15g sugar\\
9g salt\\
370g warm water\\
22g fresh yeast
\end{minipage}
\hfill
\begin{minipage}[t]{0.78\textwidth} \vspace{0pt}
	\begin{enumerate}
		\item Combine all dry ingredients in a bowl. Form a little bowl inside, add the water and dissolve the fresh yeast in it. Knead it until it has formed a smooth dough and rest for 2h at room temperature.
		\item Form it into four round pieces and let them rest for another 30min. Roll out in circles and add the toppings.
	\end{enumerate}	
\end{minipage}

\section[Pâte Sucrée]{Pâte Sucrée (for 28cm tarte ring)}
\label{patesucree}
\index{Pâte Sucrée}
\index{Tarte}
\begin{minipage}[t]{0.21\textwidth} \vspace{0pt} 
65g butter\\
45g powdered sugar\\
15g blanched, finely ground almonds\\
25g egg yolk or whole egg\\
120g flour (type 405)\\
1/4 scraped vanilla pod\\
\end{minipage}
\hfill
\begin{minipage}[t]{0.78\textwidth} \vspace{0pt}
	\begin{enumerate}
		\item Mix the soft butter with the sieved powdered sugar, vanilla and salt. Add the egg and mix until homogeneous. Add the flour and the ground almonds and knead quickly to a smooth dough. Wrap in plastic wrap and chill in the fridge for at least 1h.
		\item Take the dough out of the fridge and roll it out (roughly 2mm) on a lightly floured surface. Drape it into a 28cm tarte ring and trim the edges. Poke holes in the bottom with a fork and blind bake in the preheated oven for 15-18min.
	\end{enumerate}	
\end{minipage}

\section[Pâte Brisée]{Pâte Brisée (for 28cm tarte ring)}
\label{patebrisee}
\index{Pâte Brisée}
\index{Quiche}
\index{Tarte}
\begin{minipage}[t]{0.21\textwidth} \vspace{0pt} 
250g flour (type 550)\\
125g cold butter cubes\\
4g salt\\
1 egg\\
20g cold water
\end{minipage}
\hfill
\begin{minipage}[t]{0.78\textwidth} \vspace{0pt}
	\begin{enumerate}
		\item Mix the flour and the salt together. Then add the cold butter cubes, and break them down so the dough gets a sand-like texture.
		\item Add the egg and the cold water (as much as needed), and knead to a uniform dough. Put it in the fridge and let it rest until further use.
	\end{enumerate}	
\end{minipage}

\section{Sandwich Bread}
\index{Sandwich Bread}
\begin{minipage}[t]{0.21\textwidth} \vspace{0pt} 
175ml water\\
125ml milk\\
1 pack dry yeast\\
440g flour (type 550)\\
8g salt\\
21g sugar\\
3 tsp butter
\end{minipage}
\hfill
\begin{minipage}[t]{0.78\textwidth} \vspace{0pt}
	\begin{enumerate}
		\item Combine the dry ingredients, dissolve the yeast in the milk and water mixture and add it to the dry mix.
		\item Add the butter, knead until it is a smooth dough and let rest for 2h.
		\item Roll out in rectangle, roll up and place in buttered loaf pan. Let rise again for 45min and bake in 175° oven for 35min.
	\end{enumerate}	
\end{minipage}

\section{Pasta Dough}
\index{Pasta}
\begin{minipage}[t]{0.21\textwidth} \vspace{0pt} 
200g typo 00 flour\\
175g egg yolks\\
1 tsp salt
\end{minipage}
\hfill
\begin{minipage}[t]{0.78\textwidth} \vspace{0pt}
	\begin{enumerate}
		\item Spread flour on surface and form a little mould inside. Add the eggs and the salt in the middle and mix with a fork.
		\item Start pushing flour in the middle and mix until a cohesive smooth dough has formed.
		\item Wrap dough in plastic wrap, let chill at room temperature for 1h and then process further.
	\end{enumerate}	
\end{minipage}

\section{Ciabatta}
\index{Ciabatta}
\begin{minipage}[t]{0.21\textwidth} \vspace{0pt} 
400g flour (type 405)\\
10.5g fresh yeast\\
270g water\\
1 tsp salt\\
1 tsp sugar\\
\end{minipage}
\hfill
\begin{minipage}[t]{0.78\textwidth} \vspace{0pt}
	\begin{enumerate}
		\item Mix  the flour, salt and sugar together. Then break up the yeast in the lukewarm water, and add the mixture to the dry ingredients until they form a smooth sough after around 5min of kneading. Then let the dough rest for 2h covered at room temperature.
		\item Fold the dough on a slightly floured work surface into a tight package, and cover with a kitchen towel. Let it rest for another 30min. Preheat the oven to 220°C with convection. Now split up the dough into two loaves, roll tightly and let them rest covered for the final 15min.
		\item Take the loaves, bake them on an oven pan or a pizza stone for 22min until lightly brown and let cool to serve.
	\end{enumerate}	
\end{minipage}

\section{Challah Bread}
\index{Challah Bread}
\index{Hefezopf}
\begin{minipage}[t]{0.21\textwidth} \vspace{0pt}
360ml lukewarm water\\
21g fresh yeast\\
1 tbsp sugar\\
1 egg\\
3 egg yolks\\
120g honey\\
2 tbsp neutral oil\\
2 tsp salt\\
750g flour (type 405)\\
1 egg (egg-wash)
\end{minipage}
\hfill
\begin{minipage}[t]{0.78\textwidth} \vspace{0pt}
	\begin{enumerate}
    	\item Dissolve yeast and sugar in lukewarm water, let stand until a bit bubbly. Mix in remaining water, egg, egg yolks, honey, oil, and salt, and stir well. Gradually add the flour and mix with a machine or stand mixer until a smooth dough forms.
    	\item Let the dough rise in oiled bowl for 1h until it is doubled in volume. "Punch" dough down, tightly shape into five braids, and let rise again for about 30min. Preheat the oven to 160° with air.
    	\item Take the strands and braid them together in the challah pattern, brush the loaf with egg wash, sprinkle with seeds, and then pre-bake in the oven for 20min. Take it out, brush with egg-wash again and finish off in the oven another 20min.
	\end{enumerate}	
\end{minipage}

\chapter{Sweet Baking}
\section{Croissants}
\index{Croissants}
\begin{minipage}[t]{0.21\textwidth} \vspace{0pt} 
250ml warm water\\
30g fresh yeast\\
500g flour (type 550)\\
12g salt\\
50g sugar\\
100g softened butter\\
250g butter (beurrage)\\
\end{minipage}
\hfill
\begin{minipage}[t]{0.78\textwidth} \vspace{0pt}
	\begin{enumerate}
		\item Combine all ingredients except the butter for the beurrage to a smooth dough, and let it rest over-night. Cut the 250g butter in 1cm slices, lay out on parchment paper and place another parchment paper on top. Carefully roll out with a rolling pin until it becomes a uniform slab of putter. Also put it in the fridge over-night.
		\item Roll out the dough, place the beurrage-slab inside and fold the edges together that butter is fully enclosed in dough. Make a make tour double fold, then a tour simple fold. If necessary in case the dough is too soft or gets too sticky, you can also let the dough rest for 30min in the fridge in between folds. After both folds are performed, let it rest for 2h in the fridge.
		\item Roll out the pastry dough to 0.5cm thickness, then cut into triangles and shape the croissants. Brush them with egg-wash and let them rest for another 2h. After resting, brush them with egg-wash again and directly bake them for 18min at 190°C.
	\end{enumerate}	
\end{minipage}


\section{Biscuit Cake}
\index{Biscuit Cake}
\begin{minipage}[t]{0.21\textwidth} \vspace{0pt} 
3 eggs (room temp)\\
125g sugar\\
salt\\
90g flour (type 550)\\
1 tbsp baking soda
\end{minipage}
\hfill
\begin{minipage}[t]{0.78\textwidth} \vspace{0pt}
	\begin{enumerate}
		\item Preheat the oven to 175°C and put baking sheet in the spring-form pan.
		\item Separate the eggs, and whip the egg whites to stiff peaks while slowly adding the sugar and a pinch of salt. Then add the egg yolks and fold them into the whites, while sifting the flour into the mixture. Fold the dough with a spatula until it is homogeneous.
		\item Add dough to the spring-form and bake for 15-20min.
	\end{enumerate}	
\end{minipage}

\section{Tarte au Citron}
\index{Tarte au Citron}
\index{Tarte}
\begin{minipage}[t]{0.21\textwidth} \vspace{0pt} 
1 recipe \nameref{patesucree}\\
125g lemon juice\\
125g sugar\\
150g eggs (around 3)\\
150g butter\\
1-2 tbsp. finely grated lemon zest\\
\end{minipage}
\hfill
\begin{minipage}[t]{0.78\textwidth} \vspace{0pt}
	\begin{enumerate}
		\item For the crémeux au citron, zest one lemon and set aside. Squeeze the juice from the lemons. In a small pot, whisk the eggs with sugar until smooth. Add the lemon juice and zest, and mix well.
		\item Heat the mixture over low to medium heat while stirring constantly with a whisk, until it reaches 82-84 °C. Immediately remove from heat and strain through a fine sieve into a bowl to remove any solid particles from the eggs and lemon.
		\item Allow the mixture to cool to 35-40 °C, checking with a thermometer. Cut the cold butter into cubes and blend it into the lemon-egg cream with an immersion blender until no butter pieces are visible. Chill for about 1h in the refrigerator.
		\item Preheat the oven to 170°C. Roll out the Pâte sucrée to about 2mm thick on a lightly floured surface, forming a round shape. Line a tart ring or tart pan with the dough, trimming any excess edges. Prick the bottom of the dough with a fork.
		\item Blind bake in the preheated oven for about 15-17 minutes until golden brown. Remove from the oven and allow the tart shell to cool.
		\item Fill the tart shell generously with the crémeux au citron and smooth the top with a spatula, extending it slightly over the edges. Cover the lemon tart with a cake dome and refrigerate for at least 3 hours.
		\end{enumerate}
	\end{minipage}

\section{Pumpkin Pie}
\index{Pumpkin Pie}
\begin{minipage}[t]{0.21\textwidth} \vspace{0pt} 
1 recipe \nameref{patebrisee}\\
2 eggs\\
90g brown sugar\\
340g unsweetened condensed milk 7.5\% fat\\
450g pumpkin purée\\
2tsp pumpkin spice\\
1/4tsp ground ginger\\
1/2tsp salt\\
2tsp cornstarch\\
\end{minipage}
\hfill
\begin{minipage}[t]{0.78\textwidth} \vspace{0pt}
	\begin{enumerate}
		\item Mix flour, sugar and salt. Rub in cold butter, add ice water and form a smooth shortcrust. Shape into a ball, wrap and chill 30min.
		\item Preheat oven to 200°C. Whisk eggs and sugar, then stir in pumpkin purée, condensed milk, cornstarch and spices to a smooth filling.
		\item Roll dough slightly larger than the tart pan, line pan and trim excess. Pour in the pumpkin filling.
		\item Bake on the lowest rack 10 min at 200°C, reduce to 175°C and bake another 30-40min until the centre is slightly set. Let rest in the turned-off oven (door ajar) 1h or serve warm.
	\end{enumerate}	
\end{minipage}

\section[Chocolate Chip Cookies]{Chocolate Chip Cookies (about 15 cookies)}
\index{Chocolate Chip Cookies}
\index{Cookies}
\begin{minipage}[t]{0.21\textwidth} \vspace{0pt} 
110g butter\\
60g brown sugar\\
40g white sugar\\
1 egg\\
225g flour (type 550)\\
1/2 tsp baking powder\\
150g chocolate
\end{minipage}
\hfill
\begin{minipage}[t]{0.78\textwidth} \vspace{0pt}
	\begin{enumerate}
		\item Mix the melted butter with the sugar. Then add the flour and the baking powder and mix well. Finish by adding the chopped chocolate bars or chocolate chips.
		\item Bake in the oven for around 10-13min at 170°C with air.
	\end{enumerate}	
\end{minipage}

\section[Crêpes]{Crêpes (around 6-8 big crêpes)}
\index{Crêpes}
\begin{minipage}[t]{0.21\textwidth} \vspace{0pt}
160g flour\\
2 eggs\\
250ml milk\\
1 tsp butter
\end{minipage}
\hfill
\begin{minipage}[t]{0.78\textwidth} \vspace{0pt}
	\begin{enumerate}
		\item Mix the dry ingredients together. Then add the eggs and the milk, and whisk into an homogeneous dough. Let the dough rest for 15min and, if necessary, adjust the consistency with milk or flour.
		\item Heat a non-stick pan with little oil and bake the crêpes until golden brown.
	\end{enumerate}	
\end{minipage}

\section[Waffles]{Waffles (about 6 waffles)}
\index{Waffles}
\begin{minipage}[t]{0.21\textwidth} \vspace{0pt}
125g soft butter\\
100g sugar\\
1 pck vanilla sugar\\
3 eggs\\
250g flour (type 550)\\
200ml milk\\
1 tsp baking powder\\
salt
\end{minipage}
\hfill
\begin{minipage}[t]{0.78\textwidth} \vspace{0pt}
	\begin{enumerate}
		\item Mix all the ingredients together to a homogeneous dough and let it rest for about 10min. After resting, add more milk/flour if needed to adjust the consistency.
		\item Bake in a waffle iron for around 4-5min until golden brown.
	\end{enumerate}	
\end{minipage}

\section{Pancakes}
\index{Pancakes}
\begin{minipage}[t]{0.21\textwidth} \vspace{0pt}
220g flour\\
30g sugar\\
240ml milk\\
60g butter \textit{(optional)}\\
1 pck baking powder
\end{minipage}
\hfill
\begin{minipage}[t]{0.78\textwidth} \vspace{0pt}
	\begin{enumerate}
		\item Mix the dry ingredients together.
		\item Whisk the eggs and the milk until smooth and slowly add the powder ingredients.
		\textit{Optional: Melt the butter and add to the mixture as well.}
		\item Heat a non-stick pan at low to medium heat and bake the pancakes until golden brown on both sides.
	\end{enumerate}	
\end{minipage}

\section[Blueberry Muffins]{Blueberry Muffins (about 10-12 muffins)}
\index{Muffins}
\index{Blueberry Muffins}
\begin{minipage}[t]{0.21\textwidth} \vspace{0pt}
1 egg\\
90g powdered sugar\\
120ml milk\\
80g melted butter\\
190g flour (type 550)\\
8g baking powder\\
140g blueberries
\end{minipage}
\hfill
\begin{minipage}[t]{0.78\textwidth} \vspace{0pt}
	\begin{enumerate}
		\item Mix the dry ingredients (flour, baking powder and salt). Whisk the egg and the powdered sugar, then add the milk and the melted butter. Gradually sift in the powdered ingredients and mix to a dough.
		\item Preheat the oven to 220°C with air.
		\item Bake the Muffins for 5min, then reduce the temperature to 180°C and bake the remaining 20min.
	\end{enumerate}	
\end{minipage}

\section{Cinnamon Rolls}
\index{Cinnamon Rolls}
\begin{minipage}[t]{0.21\textwidth} \vspace{0pt}
360g flour (type 550)\\
60g white sugar\\
1/2 tsp salt\\
180ml milk\\
43g butter\\
1/2 block fresh yeast\\
1 large egg\\

45g soft butter\\
80g brown sugar\\
1 tbsp ground cinnamon
\end{minipage}
\hfill
\begin{minipage}[t]{0.78\textwidth} \vspace{0pt}
	\begin{enumerate}
		\item Preheat the oven to 190°C without air. Whisk the flour, sugar and salt together into a large bowl and set aside.
		\item Heat up the milk and the butter in a small pot, and let cool down to room temperature. Add the fresh yeast to the milk mixture, and then add the mixture and the egg to the dry ingredients. Knead until smooth and let it rest for 10min.
		\item Transfer the dough to the work surface and roll out to a rectangle, roughly 60cm x 80cm in size. Spread the soft butter on top of the dough, mix the brown sugar and cinnamon together and then evenly sprinkle it on top of the dough.
		\item Tightly roll the dough upwards, and cut the dough roll into pieces with roughly 4cm width. Place them upright onto a baking tray and bake in the oven for 24-27 minutes or until golden brown.
	\end{enumerate}	
\end{minipage}

\section{Almond Macarons}
\index{Almond Macarons}
\index{Christmas cookies}
\begin{minipage}[t]{0.21\textwidth} \vspace{0pt}
300g ground almonds\\
7 drops bitter almond extract\\
300g sugar\\
2 egg whites\\
1 pinch salt\\
a handful whole almonds for decoration
\end{minipage}
\hfill
\begin{minipage}[t]{0.78\textwidth} \vspace{0pt}
	\begin{enumerate}
		\item Mix the ground almonds with the bitter almond extract and the sugar in a bowl until well combined.
		\item Whip the egg whites with a pinch of salt to stiff peaks, then gently fold them into the almond mixture until everything is evenly moistened.
		\item Using a teaspoon, place small mounds of the dough with a good distance between them onto a baking sheet lined with parchment paper. Top each mound with a whole almond and press it in lightly.
		\item Bake the almond macaroons in a preheated oven at 130°C for about 30 minutes.
	\end{enumerate}	
\end{minipage}

\section{Gingerbread Cookies}
\index{Gingerbread Cookies}
\index{Christmas cookies}
\begin{minipage}[t]{0.21\textwidth} \vspace{0pt} 
300g flour\\
1 tsp cinnamon\\
0.5 tsp ground ginger\\
1 pinch nutmeg\\
1 pinch salt\\
1 tsp baking soda\\
180g butter\\
120g sugar\\
100g molasses\\
1 egg
\end{minipage}
\hfill
\begin{minipage}[t]{0.78\textwidth} \vspace{0pt}
	\begin{enumerate}
		\item Mix flour, spices and baking soda in a bowl. Gently heat butter, sugar and molasses in a pot until everything dissolves, then let cool slightly.
		\item Carefully pour into the flour mixture, add the egg and mix thoroughly with a dough hook. Roll the still-warm dough out to about 0.5 cm thickness. If it's too sticky, roll it between cling film and chill for about two hours. Preheat the oven to 175°C shortly before the end of the chilling time.
		\item Cut out shapes, re-roll scraps and cut more cookies. Bake the gingerbread for about ten minutes. Decorate with icing or decorations as desired after baking.
	\end{enumerate}	
\end{minipage}

\section{Macadamia White Chocolate Kisses}
\index{Macadamia White Chocolate Kisses}
\index{Christmas cookies}
\begin{minipage}[t]{0.21\textwidth} \vspace{0pt} 
130g flour (type 405)\\
40g cornstarch\\
30g powdered sugar\\
1 vanilla bean\\
120g cold butter\\
50g unsalted macadamia nuts, chopped\\
100g white chocolate (or dark chocolate)
\end{minipage}
\hfill
\begin{minipage}[t]{0.78\textwidth} \vspace{0pt}
	\begin{enumerate}
		\item Preheat the oven to 175°C. Line a baking sheet with parchment paper and set aside. Finely chop the macadamia nuts.
		\item In a large bowl, mix the flour, cornstarch, and powdered sugar. Scrape the seeds from the vanilla bean and add them along with the cold butter (in small pieces). Quickly knead everything into a smooth dough. Add the chopped macadamia nuts and knead them in. Form 24 balls from the dough (slightly smaller than a walnut) and place them on the prepared sheet. Press a small indentation into each cookie with the blunt end of a wooden spoon and then bake for 10-12 minutes. Remove from the oven, let cool briefly on the sheet, and then transfer to a wire rack to cool completely.
		\item Roughly chop the chocolate and melt it carefully - either in a bowl over a pot of simmering water or in the microwave, stirring until smooth. Let the chocolate cool slightly so it's not too runny, then fill the indentations of the cookies with a spoon or using a piping bag and let it set.
	\end{enumerate}	
\end{minipage}

\renewcommand{\indexname}{RECIPES A-Z}
\printindex

\end{document}